\documentclass[
]{jss}

\usepackage[utf8]{inputenc}

\providecommand{\tightlist}{%
  \setlength{\itemsep}{0pt}\setlength{\parskip}{0pt}}

\author{
Noelia Rico\\Department of Computer Science, University of Oviedo \And Irene Díaz\\Department of Computer Science, University of Oviedo
}
\title{Aggregating Ranking Rules in R: The Package \pkg{consensus}}

\Plainauthor{Noelia Rico, Irene Díaz}
\Plaintitle{Aggregating Ranking Rules in R: The Package consensus}
\Shorttitle{\pkg{consensus}: Ranking Rules Aggregation}

\Abstract{
The abstract of the article.
}

\Keywords{aggregation, ranking rules, social choice theory, \proglang{R}}
\Plainkeywords{aggregation, ranking rules, social choice theory, R}

%% publication information
%% \Volume{50}
%% \Issue{9}
%% \Month{June}
%% \Year{2012}
%% \Submitdate{}
%% \Acceptdate{2012-06-04}

\Address{
    Noelia Rico\\
  Department of Computer Science, University of Oviedo\\
  First line Second line\\
  E-mail: \email{noeliarico@uniovi.es}\\
  URL: \url{http://rstudio.com}\\~\\
    }


% Pandoc header

\usepackage{amsmath}

\begin{document}

\hypertarget{introduction}{%
\section{Introduction}\label{introduction}}

This template demonstrates some of the basic latex you'll need to know
to create a JSS article.

\hypertarget{code-formatting}{%
\subsection{Code formatting}\label{code-formatting}}

Don't use markdown, instead use the more precise latex commands:

\begin{itemize}
\item
  \proglang{Java}
\item
  \pkg{plyr}
\item
  \code{print("abc")}
\end{itemize}

\hypertarget{r-code}{%
\section{R code}\label{r-code}}

Can be inserted in regular R markdown blocks.

\begin{CodeChunk}

\begin{CodeInput}
R> x <- 1:10
R> x
\end{CodeInput}

\begin{CodeOutput}
 [1]  1  2  3  4  5  6  7  8  9 10
\end{CodeOutput}
\end{CodeChunk}



\end{document}

